This document details the design and construction of the EEESeaBoat, developed to scope out a simulated terrain filled with electronic ducks. Leveraging principles of circuit design, digital logic and programming, the objective was to design a robust and compact rover with relevant sensors. The primary aim was to be able to approach ducks, analyse them and provide feedback on their species and names.

Beyond detection, the project involved designing efficient code to control the rover's movement using an app, thus integrating concepts of low-latency wireless control, web interfaces and the integrated Wi-Fi module of a microcontroller. The fundamental design emphasizes cost efficiency and manoeuvrability within environmental constraints as real-world engineers would consider.

We abided by a structured process to ensure all design criteria and deadlines were met efficiently. The first phase was planning and research involved analysing components using theory and calculating appropriate values. The second phase was the design process which involved evaluating analogue against digital signal processing methods for detection systems and extracting the best ideas to ensure all criteria were met for the rover. Following this, we designed an interface for real-time user interaction and data analysis.

Finally, the hardware and software components were intertwined to execute the construction and movement of the final rover.

Testing demonstrated reliable detection of all 4 duck species as displayed on the interface with clear frequency measurements and name decoding within a total cost of roughly £45, highlighting the group's frugality.