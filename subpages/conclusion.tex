On balance, there is a clear establishment of the fact that a practical project will always have a multitude of avenues to go about handling it. Though analogue methods seem less alien given the theory covered this year, thorough research and discussion led the group to believe that digital methods supersede in terms of efficiency and scalability.

With society accelerating towards a singularity of a ubiquitous realm, electrical, electronic and information engineering continue to have undeniable impacts on our world. As demonstrated in this project, detection systems require thorough design and acknowledgement of requirements and restrictions, which is a skill needed even in real-world projects involving search and rescue mechanisms or studying species in unchartered terrain. To add contextual relevance onto this project, we can consider swarm robotics - a recently emerging field of technological research aiming to produce a network of robots which mimic the herd mentality of natural swarms such as insects. By having multiple robots coordinating and working simultaneously to complete tasks, efficiency soars without the need for one entity to control everything. In a similar sense, our project, beyond the task expectations, could have involved multiple rovers detecting different waves, but all communicating in parallel to receive and analyse data faster.

From a focussed perspective, this project utilized a range of hardware and software components. The hardware included operational amplifiers, diodes, oscilloscopes, microcontroller and a range of resistors and capacitors. Meanwhile, the software incorporated full-stack development using the Arduino as the server, and the interface as the front-end, while team skills in LTspice and 3D printing with Fusion360 were simultaneously developed.

Reflecting, despite the absences and reduced team size, members communicated well, helped explain complex concepts such as UART protocol to each other and ensured all strengths were exploited. Regardless of task allocation, the team ensured all members had a fair understanding of all sections such that everyone had a holistic view of this group project. With knowledge of digital electronics, computer architecture and circuit design, this first year rover project was successful, abided by the guidelines and provided an opportunity to place theory into a practical domain under time constraints.